\documentclass[11pt]{article}

\begin{abstract}

Road injuries are the leading cause of unnatural deaths worldwide and represent an important share of the external cost related to private transportation.  Vehicle speed is strongly associated with the probability and severity of road accidents and, as a result, cities have been experimenting with more stringent speed limits and improving enforcement on urban roads.  This paper assesses the outcomes of a set of policies that changed traffic speed limits in a major city of the developing world.  We exploit the temporal and spatial heterogeneity of policy adoption to examine the effectiveness and impacts of enforcement using measurements of fatal and nonfatal traffic accidents, traffic tickets issued by monitoring cameras, and a repeat sample of real-time trip durations for a representative sample of travelers before and after the new regulations.  Our results indicate that speed limit reductions induced a 27.5% reduction in total accident costs on treated road segments, resulting in important reductions in both fatal (21.4%) and nonfatal (34.3%) accidents while not affecting traffic volume.  We find that speed limit increases on major urban highways reduced the travel time of the average trip on those roads by 6%.  We compare the social costs and benefits of a speed limit reduction, estimating that the welfare gains resulting from the reduction in road accidents are at least 46% larger than the welfare losses resulting from longer commuting times.  We find that camera-based enforcement increased the benefits.  The study also reveals find that the losses in commuting time are increasing in individual income and the benefits of accident reductions accrue largely lower income pedestrians and motorcyclists, indicating that speed limit reductions have an important progressive welfare impact in developing country cities. 

\end{abstract}