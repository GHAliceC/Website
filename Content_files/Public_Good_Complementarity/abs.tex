\documentclass[11pt]{article}

\begin{abstract}

public goods in isolation, when in fact there may be strong complementarities between them.  This study examines the implications of public goods complementarities for economic valuation and efficient public investment, using the setting of public safety and open space in inner cities. Cross-sectional, difference-in-difference, and instrumental-variable estimates from Chicago, New York, and Philadelphia all indicate that local crime lowers the amenity value of public parks to nearby residents. Public safety improvements “unlock” the value of open-space amenities, and could raise the value that properties receive from adjacent parks from $22 billion to $31 billion in those three cities. Ignoring these complementarities risks over-estimating benefits in dangerous areas, under-estimating benefits in poor areas or conflating reduced amenity value with the preferences of local populations, and under-estimating benefits overall. While safety is more fundamental in a hierarchy of amenities, open spaces are not a luxury.
</div>

\end{abstract}