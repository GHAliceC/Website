\documentclass{article}
\usepackage{siunitx}
\usepackage{environ}
\usepackage[acronym, toc, nonumberlist]{glossaries}
\title{Title of Document}
\author{Name of Author}

\newwrite\absfile
\NewEnviron{Abstract}{
   \immediate\openout\absfile=\jobname.abs
   \begingroup
      \def\tmp{}\abstractdefs 
      \expandafter\stripdollars\BODY$\relax
      \immediate\write\absfile{\tmp}%
   \endgroup
   \abstract\BODY\endabstract
}
\long\def\stripdollars#1${\edef\tmp{\tmp#1}\stripdollarsA}
\def\stripdollarsA{\futurelet\next\stripdollarsB}
\def\stripdollarsB{\ifx\next\relax\else \expandafter\stripdollars\fi}

\def\abstractdefs{
   \def~{ }
   \def\gls##1{##1}
   \def\SI##1##2{##1 ##2}
   \def\second{s}
   \def\par{^^J}
   \def\TeX{TeX}
}

\newcommand{\texttest}{TEST}
\newacronym{BLSTM}{BLSTM}{bidirectional long short term memory network}

\begin{document}
\maketitle

\begin{Abstract}
A growing literature seeks to evaluate the resilience of transportation infrastructure in developing country cities and to target investments in infrastructure improvements.  Much of this literature focuses on improved design, materials, and costs of resilient infrastructure, but relatively little is known about the economic benefits associated with reducing the impacts of weather events in a transportation system.  This study presents some of the first estimates of the economic benefits associated with travel delays due to flood events in a developing country city.  Our empirical strategy makes use of repeat observations of trip durations from google's traffic API to separately identify the general effects of precipitation events and the specific effects of delays that result specifically from infrastructure flooding.  Our results suggest that the total economic effect of flood-induced travel delays in the city of Sao Paulo is 455 million BRL per year, which is roughly 9.6\% of the city's GDP. 
\end{Abstract}

\end{document}