\documentclass[11pt]{article}

\begin{abstract}

Housing discrimination is illegal.  However, paired-tester audit experiments have revealed evidence of discrimination in the interactions between potential buyers and realtors, raising concern about whether certain groups are systematically excluded from the beneficial effects of healthy neighborhoods. Using data from HUD's most recent Housing Discrimination Study and micro-level data on key attributes of neighborhoods in 28 US cities, we find strong evidence of discrimination in the characteristics of neighborhoods towards which individuals are steered.  Conditional upon the characteristics of the house suggested by the audit tester, minorities are significantly more likely to be steered towards neighborhoods with less economic opportunity and greater exposures to crime and local pollutants.  We find that holding locational preferences or income constant, discriminatory steering alone may contribute substantially to the disproportionate number of minority households found in high poverty neighborhoods in the United States.  The steering effect is also large enough to fully explain the differential in proximity to Superfund sites among African American mothers.  These results have important implications for studies of 'neighborhood effects' and confirm an important mechanism underlying observed correlations between race and pollution in the environmental justice literature.  Our results also suggest that the basic utility maximization assumptions underlying hedonic and residential sorting models may often be violated, resulting in an important distortion in the provision of local public goods. 

\end{abstract}