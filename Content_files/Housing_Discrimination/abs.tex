\documentclass[11pt]{article}

\begin{abstract}

Housing discrimination is illegal.  Despite this fact, paired-tester audit experiments have revealed evidence of discrimination in the interactions between potential buyers and realtors, although the most blatant forms of discrimination (e.g., refusal to show units) has declined over the last 30 years.  We explore a new dimension of housing discrimination.  Using data from HUD's most recent Housing Discrimination Study combined with novel spatial attribute data, we find strong evidence of discrimination in the characteristics of neighborhoods towards which individuals are directed.  Conditional upon the characteristics of the unit suggested by the audit tester, minorities are significantly more likely to be "steered'' towards neighborhoods with less economic opportunity and greater exposures to local pollutants.  These results have important implications for studies of neighborhood effects.  In particular, they suggest an important and understudied mechanism underlying observed correlations between race and pollution described in the environmental justice literature.  Results also suggest that the basic utility maximization assumptions underlying hedonic and residential sorting models may often be violated. 

\end{abstract}